\section{Технологическая часть}
В данной части рассматривается выбор средств реализации, описывается структура классов программы и приводится интерфейс программного обеспечения.

\subsection{Средства реализации}

Для написания данного курсового проекта был выбран язык C++~\cite{cpp-lang}.
Выбор данного языка программирования обусловлен следующим образом:
\begin{itemize}
	\item поддерживает объектно-ориентированную модель разработки, что позволяет структурировать программу и дает возможность эффективного написания качественного программного обеспечения;
	\item позволяет эффективно использовать ресурсы системы благодаря широкому набору функций и классов из стандартной библиотеки;
	\item обладает высокими показателями вычислительной производительности, а так как требуется быстродействие задач генерации реалистичных изображений, то язык C++ необходим.
\end{itemize}

В качестве среды разработки был использован Qt Creator~\cite{qt-creator}. 
Он обладает всем необходимым функционалом для написания, профилирования и отладки программ, а также создания графического пользовательского интерфейса.
Данная среда поставляется с фреймворком Qt~\cite{qt-framefork}, который содержит в себе все необходимые средства, позволяющие работать непосредственно с пикселями изображения.
Для упрощения сборки проекта программного обеспечения использовалась утилита qmake~\cite{qmake}.

\subsection{Сведения о модулях программы}

\subsection{Реализация алгоритмов}

%В листингах \ref{lst:tf_alg} -- \ref{lst:multy_tf_alg} приведены реализации алгоритма выделения наиболее информативных терминов для каждого документа.

%\lstinputlisting[label=lst:multy_tf_alg,caption=Функция работы основного потока запускающего вспомогательные потоки, firstline=33,lastline=59]{../src/pthread_tf.c}

\subsection{Интерфейс программного обеспечения}

\subsection*{Вывод}

В данном разделе были выбраны средства реализации, описаны структуры классов программы, описаны модули, а также рассмотрен интерфейс программы
