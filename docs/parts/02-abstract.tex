\section*{\centering РЕФЕРАТ}
\addcontentsline{toc}{section}{РЕФЕРАТ}
\setcounter{page}{2}

Расчетно-пояснительная записка \pageref{LastPage} с., \totalfigures\ рис., 15 лист., 14 ист., 1 прил.

КОМПЬЮТЕРНАЯ ГРАФИКА, АЛГОРИТМЫ УДАЛЕНИЯ НЕВИДИМЫХ ЛИНИЙ, Z-БУФЕР, ЗАКРАСКА, ОСВЕЩЕНИЕ, СЦЕНА, ПОСТРОЕНИЕ ТЕНЕЙ.

Целью работы является разработка программы, позволяющее создавать модели многогранника с различными параметрами, изменение параметров модели.

Для визуализации сцены использовался алгоритм с Z-буфером, а для представления модели многогранника использовалась поверхностная модель, представленная в виде списка ребер.

В процессе работы были проанализированы различные алгоритмы, методы представления, закраски геометрических моделей на сцена. Выбраны наиболее подходящие технологии решения для поставленной задачи, а также разработаны алгоритмы для их программной реализации. Разработана программа, предназначенная для создания модели многогранника с различными геометрическими и спектральными параметрами.

Проведено исследование быстродействия программы при различном количестве создания объектов на сцене с фиксированным количеством ребер. 
Также с фиксированным количеством объектов, но с различным количеством боковых ребер. 
Из результатов исследования следует, что время отрисовки сцены увеличивается как при увеличении количества объектов на сцена, так и при увеличении количества боковых ребер.