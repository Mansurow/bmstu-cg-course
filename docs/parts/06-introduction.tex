\section*{\centering ВВЕДЕНИЕ}
\addcontentsline{toc}{section}{ВВЕДЕНИЕ}

Компьютерная графика --- собой совокупность методов и способов преобразования информации в графическое представление при помощи ЭВМ. Данная область предоставляет широкий спектр алгоритмов: от воссоздания художественных эффектов в компьютерных играх до построения трехмерных объектов при моделировании сложных технологических продуктов. Но алгоритмы компьютерной графики ресурсозатратны --- чем более качественное изображение требуется получить, тем больше накладных ресурсов (времени и памяти) тратится на его синтез. Это самая главная проблема при создании реалистичных изображений и динамических сцен. 

Целью данного курсового проекта является разработка редактора композиции, состоящих из многогранников, геометрические и спектральные характеристики которые задаются пользователем. Для формировании более полного представления о полученной сцене должны присутствовать возможности передвижение камеры и источника света, и изменение спектральных характеристик источника света.

Для достижения поставленной цели необходимо решить следующие задачи:
\begin{itemize}
	\item анализ существующих алгоритмов компьютерной графики, использующие для создание реалистичной модели и трехмерной сцены;
	\item выбрать наиболее подходящих алгоритмов для решения поставленной задачи;
	\item проектирование архитектуры и графического интерфейса программы;
	\item выбрать средства реализации программного обеспечения;
	\item разработка ПО и реализация выбранных алгоритмов и структур данных;
	\item провести замеры временных характеристик разработанного программного обеспечения.  
\end{itemize}