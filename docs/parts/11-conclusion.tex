\section*{\centering ЗАКЛЮЧЕНИЕ}
\addcontentsline{toc}{section}{ЗАКЛЮЧЕНИЕ}

Целью данного курсового проекта была достигнута, то есть был разработан программный продукт, позволяющий создавать и редактировать композиции из трехмерных графических многогранников. Также предоставляет широкие возможности настройки геометрических и спектральных характеристик объектов, положения камеры и положения и цвета источника освещения. Были добавлены возможности изменения ориентации и положения камеры с помощью манипуляторов клавиатура и мышь.

Для достижение цели были выполнены следующие задачи:
\begin{itemize}
	\item проведен анализ существующих алгоритмов компьютерной графики, использующие для создание реалистичной модели и трехмерной сцены;
	\item выбраны наиболее подходящих алгоритмов (алгоритмы удаления невидимых линий, методы закраски, модели освещения, алгоритмы создания динамических теней) для решения поставленной задачи ;
	\item спроектированы архитектуры и графического интерфейса программы;
	\item выбраны средства реализации программного обеспечения;
	\item разработаны ПО и реализация выбранных алгоритмов и структур данных;
	\item проведены замеры временных характеристик разработанного программного обеспечения.  
\end{itemize}

Данный программный продукт может быть использован для демонстрации спектральной и диффузной отражающей способностей различных материалов, а также для визуализации падения теней при различных условиях освещенности. 

Быстродействие низкоуровневых частей программы, отвечающих за растеризацию полигонов, может быть улучшено за счет замены их программной реализации на поддерживаемую аппаратно из интерфейсов графических движков, например, OpenGL или DirectX.

